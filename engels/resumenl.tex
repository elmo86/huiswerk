% resume.tex
% vim:set ft=tex spell:

\documentclass[10pt,letterpaper]{article}
\usepackage[letterpaper,margin=0.75in]{geometry}
\usepackage[utf8]{inputenc}
\usepackage{mdwlist}
\usepackage[T1]{fontenc}
\usepackage{textcomp}
\usepackage{tgpagella}
\pagestyle{empty}
\setlength{\tabcolsep}{0em}

% indentsection style, used for sections that aren't already in lists
% that need indentation to the level of all text in the document
\newenvironment{indentsection}[1]%
{\begin{list}{}%
	{\setlength{\leftmargin}{#1}}%
	\item[]%
}
{\end{list}}

% opposite of above; bump a section back toward the left margin
\newenvironment{unindentsection}[1]%
{\begin{list}{}%
	{\setlength{\leftmargin}{-0.5#1}}%
	\item[]%
}
{\end{list}}

% format two pieces of text, one left aligned and one right aligned
\newcommand{\headerrow}[2]
{\begin{tabular*}{\linewidth}{l@{\extracolsep{\fill}}r}
	#1 &
	#2 \\
\end{tabular*}}

% make "C++" look pretty when used in text by touching up the plus signs
\newcommand{\CPP}
{C\nolinebreak[4]\hspace{-.05em}\raisebox{.22ex}{\footnotesize\bf ++}}

% and the actual content starts here
\begin{document}

\begin{center}
{\LARGE \textbf{Paul Sohier}}

Paul Desmondsingel 166\ \ \textbullet
%\ \ Suite\ 1001\ \ \textbullet
\ \ 3069RW Rotterdam\ \ 
\ 
\\
06 198 233 70\ \ \textbullet
\ \ paul@paulsohier.nl
\end{center}

\hrule
\vspace{-0.4em}
\subsection*{Ervaringen}

\begin{itemize}
        \parskip=0.1em
        \item
          \headerrow
          {\textbf{ABB}}
          {\textbf{Rotterdam}}
          \\
          \headerrow
          {\emph{Stagair elektronica werkplaats}}
          {\emph{2006-2007}}
          \begin{itemize*}
            \item Ontwikkeling van testmethode voor frequentieregelaars
          \end{itemize*}

        \parskip=0.1em
        \item
          \headerrow
          {\textbf{Carrier Transicold}}
          {\textbf{Rotterdam}}
          \\
          \headerrow
          {\emph{Stagair reparatiewerkplaats}}
          {\emph{2008 -- 2008}}
          \begin{itemize*}
            \item Reparatie diverse type vrachtwagenkoeling
            \item Onderzoek naar diverse communicatiestandaarden 
          \end{itemize*}

	\parskip=0.1em

	\item 
	\headerrow
		{\textbf{Hoge School Rotterdam}}
		{\textbf{Rotterdam}}
	\\
	\headerrow
		{\emph{Student assistent}}
		{\emph{2010 -- 2011}}
	\begin{itemize*}
        \item Toezien op wekelijkse tentamens
        \item Beheer van aanwezige hardware en uitlenen spullen
	\end{itemize*}

        \parskip=0.1em
        \item
          \headerrow
          {\textbf{Hoge School Rotterdam}}
          {\textbf{Rotterdam}}
            \\
            \headerrow
            {\emph{Docent}}
            {\emph{2011 -- heden}}
            \begin{itemize*}
              \item Lesmateriaal geschreven voor het vak Linux (http://www.hosthuis.nl/linux/)
              \item Wekelijkse presentaties over de theorie
              \item Wekelijkse beoordelen en begeleiden van studenten
            \end{itemize*}
\end{itemize}

\hrule
\vspace{-0.4em}
\subsection*{Scholing}

\begin{itemize}
	\parskip=0.1em

        \item
          \headerrow
          {\textbf{ROC Zadkine}}
          {\textbf{Rotterdam}}
          \\
          \headerrow
          {\emph{MBO Elektronica}}
          {\emph{2004 -- 2008}}


	\item 
	\headerrow
		{\textbf{Hoge School Rotterdam}}
		{\textbf{Rotterdam}}
	\\
	\headerrow
		{\emph{Technische Informatica}}
		{\emph{2009 -- heden}}
	\begin{itemize*}
          \item Onderzoek en bouw van een 3D printer
          \item Ontwikkeling van een elektronische tentamen systeem         
	\end{itemize*}

\end{itemize}

\hrule
\vspace{-0.4em}
\subsection*{Open Source/Vrijwilligerswerk}

\begin{itemize}
	\parskip=0.1em

	\item
	\headerrow
		{\textbf{phpBB}}
		{\textbf{}}
	\\
	\headerrow
		{\emph{MODifications team member}}
		{\emph{2006 -- heden}}
	\begin{itemize*}
          \item Verantwoordelijk voor interne veiligheid
          \item Als teamleider verantwoordelijk voor validatie
          \item Software ontworpen voor snellere validatie
          \item Diverse presentaties gegeven over veiligheid en phpBB
	\end{itemize*}

        \item
          \headerrow{\textbf{Vrijwillgerswerk sportclub}}{\emph{2010 -- heden}}
 %         \\
%          \headerrow{}{\emph{2010 -- heden}}
          \begin{itemize*}
          \item Maandelijks bardienst in kantine
            \item Sociaale hygiene behaald voor de vereniging
            \item Maken van schema's voor de bardiensten in de kantine
            \item Advies over gezonde kantine voor de jeugdleden
          \end{itemize*}
\end{itemize}


\hrule
\vspace{-0.4em}
\subsection*{Technische vaardigheden}

\begin{indentsection}{\parindent}
\hyphenpenalty=1000
\begin{description*}
	\item[Talen:]
	C, \CPP, Java, JavaScript, \LaTeX, PHP, shell scripting, SQL
	\item[Open Source contributies:]
	phpBB, Debian
%        \item[Overige:] Derde op de Euroskills elektronica competitie
\end{description*}
\end{indentsection}

\end{document}
