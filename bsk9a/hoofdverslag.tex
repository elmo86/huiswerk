\hoofdstuk{Stakeholders}
Bij een fitnessbedrijf zijn er diverse stakeholders die belangrijk zijn bij het uitoefenen van het bedrijf. Wanneer je niet weet wie de stakeholders zijn is het lastig om een idee te hebben met die partijen je als ondernemer allemaal te maken hebt. 

Aangezien er bij dit fitnessbedrijf zowel particulieren als bedrijven klanten zijn, zijn dit aparte stakeholders. Naast deze twee type klanten heb je ook nog de diverse leveranciers van de fitnessapparatuur, die ook een belangrijk onderdeel zijn voor het uitoefenen van het bedrijf. Als het bedrijf ook nog een (kleine) kantine heeft, zijn naast de leveranciers van de fitnessapparatuur ook de leveranciers van drank een stakeholder.
Naast de leveranciers en klanten zijn er ook nog de buren waarmee rekening gehouden moet worden, bijvoorbeeld wanneer het bedrijf overlast veroorzaakt in de buurt. Om het bedrijf te starten of over te nemen zijn er mogelijk ook nog vergunningen nodig, waarvoor je de lokale politici nodig hebt.
Om het bedrijf draaiend te houden heb je uiteraard ook personeel nodig en dit is dus de laatste en na de klanten ook de belangrijkste stakeholder.

In totaal kom je dus op de volgende stakeholders die van belang zijn:
\begin{itemize}
        \item Klanten, zowel particulier als zakelijk
        \item Leveranciers van diverse producten en apparatuur
        \item Buren en politici
        \item Personeel
\end{itemize}

\hoofdstuk{Macro Situatie}

\paragraaf{Demografische}
Voordelen: 
\begin{itemize}
        \item Klanten bestand is al aanwezig, er zijn dus al inkomsten.
        \item Diverse soorten klanten waardoor er grotere spreiding is over de inkomstenbron.
\end{itemize}

Nadelen:
\begin{itemize}
        \item Groot verloop in klanten, mensen stoppen relatief vaak en beginnen weer opnieuw wat zorgt voor een grote administratieve last.
\end{itemize}

\paragraaf{Ecologische}
Een fitnessbedrijf heeft relatief weinig invloed op het ecologische gebied, een verstandige keuze is om bijvoorbeeld afval te scheiden en gebruik te maken van apparatuur die minder stroom verbruiken. Aangezien er echter al apparatuur aanwezig is is het lastig om te bepalen of dit zonder grote kostenpost eenvoudig aan te passen is. Aangezien het bedrijf geen hinder, gevaar, overlast of schade veroorzaakt is er in principe geen milieuvergunning nodig.

\paragraaf{Politiek}
 Om het bedrijf draaiend te houden moeten er diverse vergunningen aangevraagd worden bij de gemeente en in sommige gevallen zijn hier ook specifieke eisen aan gesteld:
\begin{itemize}
        \item Bedrijven moeten 4 weken voor begin/opengaan dit kenbaar maken aan de brandweer waarbij er gekeken wordt naar de veiligheid van het pand. Het veilig gebruik van het pand ligt in handen van de gebruiker, de brandweer kan wel veiligheidseisen stellen.
        \item Horeca vergunning: In het geval dat er drank verkocht wordt is er een drankvergunning nodig. Deze vergunning dient aangevraagd te worden bij de gemeente. Voor deze vergunning zijn ook diverse certificaten verplicht, waaronder bijvoorbeeld sociale hygiëne. Ook zitten er aan deze vergunning diverse vereisten waaraan gehouden moet worden.
\end{itemize}

Naast de benodigde vergunningen kan er ook bijvoorbeeld met de gemeente overlegd worden om scholen te stimuleren te sporten via het bedrijf. Hierbij moet dan echter wel gekeken worden naar oudere jeugd.

\paragraaf{Economische}
De vereiste investering in het begin van een fitnessbedrijf is hoog door de aanschaf van dure apparatuur. Na de aanschaf van deze apparatuur is er echter verder geen investering nodig, behalve voor vervanging van de apparatuur. Om de apparatuur te bekostigen zal dus in veel gevallen een lening benodigd zijn, die dus ook met regelmaat afbetaald moet worden. Een vast inkomen vanuit klanten is dus wel een vereiste om dit te kunnen betalen. Naast de vaste kosten voor de lening zijn er ook de kosten voor het personeel en de huur van het pand die betaald moeten worden iedere maand. Door het hoge verloop van klanten kan dit een probleem veroorzaken.

\paragraaf{Sociaal}
Veel klanten zullen naar de fitness gaan om iets te doen aan hun ongezonde leefstijl en niet specifiek om gezellig om te gaan met mensen. Het zal in veel gevallen dus weinig toevoegen binnen het bedrijf om mensen meer aan te sporen om sociale contacten op te bouwen.

\paragraaf{Technologische}
Binnen de fitness wereld gebeurd het met regelmaat dat apparatuur snel verouderd raakt en hierdoor sneller vervangen moet worden dan dat ze economische afgeschreven zijn, dit zorgt voor een extra verlies voor het fitnessbedrijf. De eigenaar moet op zo 'n moment keuzes maken of dit bijdraagt aan het bedrijf. 

\hoofdstuk{Meso Situatie}
\paragraaf{Leveranciers}
Er zijn maar een beperkt aantal leveranciers van producten die een fitnessbedrijf nodig heeft. Zodra een leverancier gekozen is zal die leverancier het fitnessbedrijf in principe als klant willen houden. Hierdoor zal de leverancier ook zijn best doen het contact goed te houden en niet direct aan klanten van de sportschool leveren, wat sowieso in deze branche niet snel zou gebeuren. Daarintegen is de kans wel groot dat ze gaan leveren aan de directe concurrenten, door het lage aanbod aan leveranciers. Hierdoor kan het bedrijf een unieke apparatuur welke het aangeschaft had bij de leverancier en gebruikt als reclame materiaal verliezen.

\paragraaf{Potenti\"{e}le toetreders}
Ondanks dat er al een hoop bedrijven zijn die fitness mogelijkheden aanbieden, komen er nog steeds een hoop bedrijven bij waardoor de concurrentie nog meer toeneemt. Hierdoor zal een start voor nieuwe bedrijven lastig zijn en zal het veel tijd kosten voordat er klanten zullen komen.

\paragraaf{Kopers}
Doordat er tegenwoordig een groot aantal fitnessbedrijven zijn in de grotere steden zullen veel mensen kijken naar welk bedrijf daadwerkelijk de beste is en wordt er veel gekeken naar prijs en service. Om ervoor te zorgen dat klanten niet naar de concurrentie gaat zullen bedrijven daadwerkelijk ervoor moeten zorgen dat de service die verleend wordt correct is. Klanten gaat het primair erom dat het goedkoop en goed van kwaliteit is, een hogere reisafstand maakt hierbij dus niet uit.

\paragraaf{Substituten}
De concurrentie in de fitnessbranche is enorm, in veel grote steden komen er steeds meer (kleine) concurrenten bij die klanten zullen proberen over te nemen van al bestaande bedrijven. Door te concurreren op prijs en service zullen ze zich voordoen als beter als andere fitnessbedrijven. Je ziet dat ze in goedkope locaties gaan zitten, zoals bijvoorbeeld oude bedrijfspanden die niet langer gebruikt/verhuurd worden, waardoor de kosten van pand omlaag gaan. Op deze manier proberen deze nieuwe bedrijven daadwerkelijk de klant te binden tegen een lagere prijs als de concurrenten die op een betere locatie zitten.

\hoofdstuk{Analyse}
De kosten zijn voor de consument de belangrijkste reden\footnote{\url{http://www.ing.nl/businessbanking/sectoren/leisure/leisure.aspx}} om naar een bepaald bedrijf te gaan die de mogelijkheid tot fitnissen aanbied. Een bedrijf zal hierop duidelijk moeten inspringen en concurrenten moeten volgen om te kijken of ze daadwerkelijk voor de service die het bedrijf bied ook echt de goedkoopste bent. Zodra een fitnessbedrijf niet meer de goedkoopste is, is de kans groot dat de klant naar de concurrent zal overstappen. Om kans te maken in de huidige markt moet je als bedrijf onderscheidend zijn met een uniek concept\footnote{\url{http://www.rabobank.nl/images/horeca_fitnesscentra_okt2011_2936115.pdf?ra_resize=yes&ra_width=800&ra_height=600&ra_toolbar=yes&ra_locationbar=yes}}, dit betekend echter wel dat de investeringen die gedaan moeten worden in het bedrijf stijgen. Wanneer een bepaald concept op zo 'n moment niet aanslaat zal het bedrijf de kosten die gemaakt zijn niet terugverdienen. Dit zorgt voor een relatief groot risico om als bedrijf iets te proberen. Doordat de concurrentie enorm is in de branche is de kans dat een bepaald concept niet aanslaat bij klanten relatief groot. De bedreiging van het bedrijf is hierdoor op dit moment relatief groot.  

Naast het toepassen van specifieke formules binnen het bedrijf moet er ook gekeken worden naar het gebruik van bijvoorbeeld sociale media om klanten te werven en te behouden. 

\hoofdstuk{Aankoop}
Aangezien de slechte staat van de fitnessbranche door de hoge concurrentie en vaak niet onderscheidende bedrijven is de overname van een fitnessbedrijf op dit moment een slecht idee. De kans dat je de investeringen die je doet in het bedrijf terug zal verdienen is minimaal en een hoop bedrijven zijn in de afgelopen tijd gestopt vanwege de slechte situatie van de branche. Een groot aantal overnames kan betekenen dat bedrijven het niet rond krijgen en kan dus een slecht teken zijn.

Ondanks dat de banken zien dat er nog steeds veel fitnessbedrijven worden gestart of overgenomen is de kans dat het bedrijf failliet gaat door de concurentie relatief groot. Mijn conclusie is dan ook om het bedrijf niet over te nemen door de te hoge risico's welke genomen moeten worden om het bedrijf winstgevend te maken. Dit is ook direct samenhangend met de inversteringen welke gedaan moeten worden in het bedrijf om klanten te binden aan het bedrijf.
        
 
