\hoofdstuk{Inleiding}
In 2008 werd besloten tot de invoering van een landelijk elektronische patienten dossier (EPD) voor het opvragen van informatie van medewerkers van zorginstellingen en zorgverleners. In het EPD zou alle informatie die artsen en zorgverleners hebben opgeslagen worden, waarbij iedere medewerker welke werkt met de patient toegang heeft tot zijn of haar gegevens. Iedere Nederlander kreeg in eind 2008 een brief toegestuurd met de vraag of ze bezwaar tegen het EPD hebben. In december 2008 hadden ongeveer 330.000 mensen bezwaar gemaakt tegen het opnemen in het EPD\cite{bib.minvws.epdbezwaar}. Een groot bezwaar was dat de privacy in het geding was wanneer er centraal gegevens opgeslagen werden en hulpverleners gemakkelijk bij de gegevens kunnen.

De toegang tot het EPD wordt centraal geregeld met een UZI\footnote{Unieke zorgverlener identificatie}-pas. In het UZI-register staat opgeslagen welke zorgverleners er zijn, met hun naam en functie beschrijving, en waartoe ze toegang hebben. Voor een veilige communicatie wordt er gebruik gemaakt van certificaten die uitgegeven worden door de overheid, of door een door de overheid aangesteld bedrijf.

De gegevens die in het EPD opgeslagen staan worden niet bij de overheid zelf bewaard, maar bij een commercie\"{e}le partij welke de Europese aanbesteding gewonnen heeft. 
De certicaten die nodig zijn voor het EPD worden uitgegeven door de overheid onder de Staat der Nederlanden root Certification Authority (CA). Dit root CA is eigendom van de Nederlandse Staat en werd vertrouwd door alle besturingssystemen en internet browsers. De Nederlandse Staat heeft het uitgeven van Certificaten onder dit root CA uitbesteed aan diverse bedrijven, zoals bijvoorbeeld Getronics in het geval van het EPD\cite{bib.minvws.CPSUZI}.

Een groot deel van het EPD is uitbesteed aan commercie\"{e}le bedrijven waarvan de vraag is of deze daadwerkelijk als doel hebben de privacy van de personen te waarborgen of zelf primair winst te halen uit het project. 
