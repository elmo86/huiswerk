\hoofdstuk{Conclusies en aanbevelingen}
Zowel bij de overheid zelf als bij externe bedrijven gaat er het nodige mis bij ICT projecten en ICT dienstverleners in het algemeen. Bij het gebruik van commerciele bedrijven, zoals bijvoorbeeld Diginator, is er geen controle op de werking en beveiliging van het bedrijf en heeft Logius geen overzicht of de bedrijven daadwerkelijk nog voldoen aan het programma van eisen. Zodra een bedrijf eenmaal toegang tot de PKIOVerheid heeft lijkt het erop dat de overheid en daarmee met name Logius niet langer controleert of een bedrijf daadwerkelijk voldoet aan de gestelde eisen door Logius zelf. Doordat Logius de richtlijnen niet controleerd heeft Logius en dus de overheid geen idee of de bedrijven voldoen en of de infrastructuur die de overheid gebruikt voor de diensten die ze aanbieden aan de burger daadwerkelijk veilig zijn. Bij het Diginator schandaal had de overheid in eerste instantie niet door dat het probleem daadwerkelijk zo groot was dat alle overheid sites, waaronder bijvoorbeeld DigiD en de rijksdienst van het wegverkeer niet langer veilig konden communiceren met burgers. De veiligheid was niet gegarandeerd aangezien niemand kon zeggen of het certificaat wat gebruikt was daadwerkelijk een geldig en goed gecontroleerd certificaat was.

Zowel het intern beheren van het EPD binnen de overheid als het uitbesteden aan commercie\"{e}le bedrijven leveren problemen op voor de privacy en veiligheid van de data. Bij de overheid zelf wordt er op een verkeerde manier de projecten aangestuurd waardoor projecten over budget gaan en de tweede kamer hierna in veel gevallen beslist dat er bezuinigd moet worden op de projecten. Deze bezuiniging levert weer problemen op doordat het project dan gewoonweg niet volledig afgemaakt wordt en de gebruiker (Zoals bijvoorbeeld de politie) met een ICT infrastructuur komen te zitten welke niet voldoet aan de eisen die ze gesteld hebben. In het geval van het systeem bij de politie had dit uiteindelijk als gevolg dat de politietop besloot om de complete infrastructuur opnieuw vervangen moest worden naar een systeem wat wel correct werkten en wat niet het gevolg had dat agenten meer tijd kwijt waren aan administratie als aan hun dagelijkse taken.

De enige conclusie die je kan maken is dat geen van beide opties voor het beheer, het intern beheren bij de overheid en het uitbesteden naar een commercie\"{e}le partij, eigenlijk een optie is. Als we kijken naar de geschiedenis van ICT projecten gaat er zo ontzettend veel mis dat de veiligheid van het product wat gebruikt gaat worden niet gegarandeerd kan worden en dat de privacy van burgers in in het geding is. Doordat communicatie tussen de verschillende locaties (In het geval van het EPD tussen zorgverleners en de centrale opslag) beveiligd wordt met certificaten hoort het, wanneer de richtlijnen van Logius gevolgd worden, in principe beveiligd te worden. Het voorbeeld van Diginator laat hierbij zien dat dit niet het geval is en niemand heeft een idee of het bij de andere uitgevers van certificaten van de overheid wel correct zijn uitgegeven en of ze voldoen aan de richtlijnen welke gesteld zijn door Logius.

Voordat de overheid gaat beginnen aan nieuwe grote projecten moeten ze eerst de huidige ICT infrastructuur welke ze hebben op orde krijgen. Hieronder valt niet alleen de veiligheid van de diverse infrastructuur welke ze beheren of uitbesteden maar ook het controleren van de richtlijnen welke gesteld zijn om ervoor te zorgen dat bedrijven welke werk uitvoeren voor de overheid zich houden aan deze richtlijnen. Om ervoor te zorgen dat dit alles correct wordt gedaan is het verstandig om een derde partij te vragen dit te controleren en hierin te adviseren. Uiteraard dient deze derde partij geen connecties te hebben naar zowel de overheid zelf als de partij welke het werk uitvoert voor de overheid. Zodra de huidige problemen met de ICT projecten binnen de overheid zijn opgelost kan er opnieuw naar gekeken worden hoe het beste nieuwe projecten uitgevoerd kunnen worden en of deze uitbesteed kunnen worden aan een andere partij. Door de controlerende functie bij een andere partij onder te brengen wordt er voor gezorgd dat geen van beide partijen invloed kunnen hebben op de controle, wat de kwaliteit van de te leveren dienst of infrastructuur verbeterd. De overheid zelf zal uiteraard ook eigen controles moeten uitvoeren om ervoor te zorgen dat de kwaliteit en veiligheid op een niveau blijven welke voldoen aan de eisen. 

Naast de controles moet de eerste opdracht van het cree\"{e}ren van de dienst of de infrastructuur ook op een manier worden opgesteld dat er rekening gehouden wordt met de veiligheid en de uitendelijke werking van het af te leveren product. Wanneer de eerste stap in dit proces al verkeerd is, wat bijvoorbeeld gebeurd is bij de nieuwe aanbesteding van DigiD waarbij het niet langer een eis was om SMS ondersteuning te hebben voor de controle bij het inloggen, zal het uiteindelijk product ook van een kwaliteit zijn welke niet voldoet aan hetgeen het eigenlijk hoort te voldoen. Minister Opstelten is het hier echter niet\cite{bib.tweakers.opstelten} mee eens, en is van mening dat het in de eigen organisatie hoort. Hij stelt hierbij wel dat dit financieel gezien niet rendabel is. Ik ben hierbij van mening dat bij veiligheid, ook op het internet, financie\"{e}n niet het probleem moeten zijn, en dit dus niet als argument gebruikt moet worden. Uiteraard wordt dit argument nu door de minister niet gebruikt zoals ik adviseer, maar juist ertegen.

Ook moet er naast het controleren ook een draaiboek klaar liggen voor het geval er toch nog een probleem bij  \'{e}\'{e}n van de betrokken partijen waardoor de partij vervangen kan worden door een andere, zoals bijvoorbeeld een leverancier van certificaten vervangen worden door een andere leverancies. Op deze manier kan je bij problemen voorkomen dat het probleem groter wordt als daadwerkelijk nodig is door snel te reageren op problemen. Hierbij is communicatie naar alle betrokkenen, dus in veel gevallen ook de burgers, zeer belangrijk.

Voor het EPD is de overheid als we kijken naar de richtlijnen die gesteld zijn opzich goed bezig. Hierbij is niet gecontroleerd of deze richtlijnen ook daadwerkelijk gevolgd worden en of deze afdoende zijn. Er is door de overheid echter wel over nagedacht. Het helpt hierbij ook dat er algemeen bekend is wie de leveranciers zijn van bepaalde hardware en diensten, wat zorgt voor openheid en controleerbaarheid voor derden.
