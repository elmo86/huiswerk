\hoofdstuk{Certificaten}
Om ervoor te zorgen dat de gegevens welke opgevraagd worden in het EPD beveiligd over het internet verstuurd wordt maakt het EPD gebruik van certificaten. Hiermee wordt het verkeer wat over het internet gaat versleuteld en kan een ongewenst persoon niet de gegevens lezen welke verstuurd zijn. Voor ieder certificaat is er een publiek en een private deel, waarmee de server en de ontvanger de data kan encrypten en decrypten. 

Om deze manier van communiceren goed te laten werken is vertrouwen een belangrijk onderdeel. Certificaten worden door de certificate authority uitgedeeld op basis van vertrouwen bij het aanvragen van een certificaat. De aanvrager moet erop kunnen vertrouwen dat de CA alleen certificaten uitgeeft aan personen welke daadwerkelijk te vertrouwen zijn en welke ook daadwerkelijk recht hebben op een certificaat.

Om ervoor te zorgen dat de uitgave van certificaten onder het root certificaat Nederlandse Staat goed verlopen zijn hiervoor richtlijnen gemaakt door Logius\cite{bib.logius.pve}. Logius is een organisatie welke onderdeel is van het Ministerie van Binnenlandse zaken en welke publieke dienstverleners diensten levert die ervoor moeten zorgen dat de ICT-infrastructuur betrouwbaar is. Verwacht zou worden dat door het naleven van deze richtlijnen het vertrouwen in de leveranciers van de certificaten gewaarborgd zijn, aangezien slechts een beperkt aantal bedrijven daadwerkelijk voldoen aan deze richtlijnen. 

In augustus 2011 bleek echter dat dit niet het geval is\cite{bib.webwereld.diginotar1}. Bij het bedrijf Diginotar had in juli 2010 een hack plaatsgevonden en hadden de hackers toegang gekregen tot de systemen waarop certificaten gegeneerd konden worden. Het bedrijf zelf had op dat moment niet naar buiten gebracht wat er had plaatsgevonden, pas toen het in Iran bleek dat er certificaten voor onder andere gmail.com waren uitgegeven door Diginotar werd het bekend. De eerste reacties van Diginotar en het ministerie waren dat het certificaat van de Staat der Nederlanden en daarbij de CA van Diginotar welke onder de Staat der Nederlanden viel niet in handen waren gevallen door de hackers. Ook beweerde Diginotar dat alle certificaten welke waren aangemaakt door de hacker al in juli waren ingetrokken, op \'{e}\'{e}n na voor google.com\cite{bib.webwereld.diginotar2}. Maar op hetzelfde moment konden ze niet garanderen dat er niet nog meer certificaten aangemaakt waren, Diginator zelf had geen idee wat de hacker had gedaan en had ook geen lijst met certificaten welke er aangemaakt waren. Hierdoor kon niemand zeggen of er daadwerkelijk geen valse certificaten meer aanwezig waren. Op 2 september, 4 dagen na het ontdekken van het eerste certificaat laat Google weten dat ze 247 certificaten valse certificaten gevonden hebben\cite{bib.webwereld.diginator3}. Diginator kon ook op dit moment niet bevestigen dat dit waar was, en of dat er nog meer valse certificaten waren uitgegeven, maar Diginator hield wel vol dat het CA van de Staat der Nederlanden nog steeds veilig was en gewoon gebruikt kon blijven. Het ministerie besloot dit advies van diginator over te nemen en kwam dus ook met het bericht dat Diginator nog steeds te vertrouwen is en de sites beveiligd met een Diginator certificaat van de Staat der Nederlanden ook gewoon veilig zijn.

Onder andere Mozilla, Google en Microsoft besloten om de certificaten die door Diginator uitgegeven waren te markeren als gevaarlijk, wat betekend dat de normale gebruikers foutmeldingen krijgen dat de site niet langer veilig is. Mozilla wou in eerste instantie alle CA's van Diginator intrekken, maar onder druk van de Nederlandse regering besloot Mozilla dit niet te doen. Ze vertrouwden, op dat moment, dus nog expliciet de Staat der Nederlanden. Wanneer Mozilla (En andere bedrijven die browsers maken) dit niet hadden gedaan zou bijvoorbeeld DigiD niet langer werken zonder foutmelding, aangezien DigiD gebruik maakten van een certificaat uitgegeven door Diginator.

Een belangrijk onderdeel van het certificaat stelsel is vertrouwen. Diginator laat met de hack al zien dat ze hun beveiliging niet op orde hebben en dat hun communicatie en kennis over wat er gebeurd was bij de hack zeer minimaal en op sommige delen gewoonweg incorrect is. Browsers hadden al het vertrouwen opgezegt na enkel de hack, maar uitendelijk blijkt dat er veel meer mis is bij Diginator. 

Na een onderzoek door Onderzoeksbedrijf FoxIt blijkt op 3 september 2011 dat ook het certificaat van de Staat der Nederlanden misbruikt is\cite{bib.foxit}. Op dit moment besluit het ministerie dat ze geen vertrouwen meer hebben in Diginotar\cite{bib.webwereld.diginator4}. Uit het onderzoek van diginator blijkt dat er een mis is met de beveiliging van Diginator en dat ze niet voldeden aan de eisen welke Logius opgesteld had. Uit het onderzoek blijkt bijvoorbeeld dat de hackers voor lange tijd toegang gehad hebben tot de systemen van Diginator. Dit kwam mede doordat er compleet geen anti-virus software was geinstalleerd op de gebruikte computers binnen het netwerk, inclusief de server voor het generen van de certificaten. Ook werd er gebruik gemaakt van wachtwoorden welke gemakkelijk geraden konden worden. 
Het is gebruikelijk om de certificaat servers buiten het netwerk te houden zouden deze niet via het internet bereikbaar zijn waardoor het hacken van een (deel) van het netwerk niet schadelijk is en een hacker geen toegang krijgt tot de servers en sleutels waarmee certificaten gegenereerd worden.

Het is duidelijk dat Diginator het vertrouwen zwaar beschadigd heeft. Om deze reden heeft de Opta ook besloten dat Diginator niet langer certificaten uit mag geven en heeft het moederbedrijf van Diginator het failisement van Diginator aangevraagd.

\hoofdstuk{Website veiligheid}
Tegenwoordig heeft ieder bedrijf en iedere gemeente een eigen website. Wanneer je begint met het ontwikkelen van webapplicaties is \'{e}\'{e}n van de belangerijke onderdelen de veiligheid van de websites. Helaas is uit proeven van Webwereld.nl, Geenstijl en Nu.nl gebleken dat gemeente sites niet in alle gevallen veilig zijn\cite{bib.tweakers.gemeente}. In dit geval spreken we niet over problemen zoals eerder beschreven met certificaten, maar met veiligheids problemen op de website zelf. Hierbij kon een onbevoegd persoon toegang krijgen tot data waartoe hij geen toegang hoorde te hebben, zoals bijvoorbeeld backups van de websites en prive gegevens van ambtenaren. Ook kon in sommige gevallen toegang tot gegevens van burgers verkregen worden doordat je als onbevoegd persoon toegang kon krijgen tot DigiD van burgers. 

Gemeente reageerde hierbij in sommige gevallen op door de onderzoeksjournalist brieven van een advocaat te sturen met het verzoek het niet te publiceren, maar te rectificeren dat de gemeente niet fout zat. Ze wouden hierbij niet toegeven dat de gemeente fout zat.

\hoofdstuk{ICT projecten}
Ook ICT projecten lopen niet altijd even goed bij de overheid. Een voorbeeld hiervan zijn de ICT systemen bij de politie\cite{bib.tweakers.politie}. Dit systeem zorgde voor de politie voor zulk slecht werk klimaat dat het niet meer te doen was om met de systemen te werken. Het systeem was specifiek voor de Nederlandse politie ontworpen, maar hierbij was geen rekening gehouden met de gebruikers welke met het systeem moesten gaan werken en was er ook niet gekeken naar budget. Hierdoor kwam er aan het eind van het project een product uit wat totaal niet te gebruiken was voor de politie en wat ook nog eens ver over het budget gegaan was. 

Om het probleem bij de politie op te lossen is er weer besloten om een geheel nieuw systeem voor de politie te ontwikkelen. De vraag is hierbij natuurlijk of dit nieuwe systeem niet dezelfde soort problemen gaat krijgen als het huidige systeem, wat om eenzelfde reden is ontwikkeld.


