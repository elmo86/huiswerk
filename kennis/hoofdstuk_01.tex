\chapter{Fablab}
\section{Wat is fablab}
fablab is begonnen op MIT, waarbij het doel van een fablab is om een locatie aan te bieden waar iedereen die wilt tegen materiaalkosten kan werken aan zijn projecten. De enige eis van een fablab is dat hetgeen wat gemaakt wordt door een gebruiker daadwerkelijk ook opensource is en een link op de website van dat specifieke fablab komt te staan. 
Fablabs kunnen opgezet worden met een relatief laag bedrag, ongeveer €25.000, aan machines waarmee ongeveer alles gedaan kan worden wat ontworpen moet worden. 
Volgens fablab.nl is een fablab als volgt:
\begin{quote}
Het blijft vaak voor mensen onduidelijk wat een FabLab precies inhoudt. Je kan er op een hele abstracte manier over praten, van “geavanceerde digitale produktie werkplaats” tot “creatieve leeromgeving” en zelfs “een soort kerk waar mensen samen komen ter ere van innovatie”, dat zijn allemaal mooie omschrijvingen die goed passen bij de processen die zich afspelen in een FabLab. Maar het ontneemt soms een beetje het zicht op wat een FabLab nu concreet inhoudt.


Een FabLab is een ruimte waarin zich een aantal machines bevinden. Dit zijn eenvoudige digitaal aangestuurde productiemachines. Deze set van machines is in alle FabLabs min of meer hetzelfde. Omdat de machines aangestuurd worden via computers, en de machines in elk FabLab hetzelfde zijn, is het heel makkelijk om ontwerpen te delen. Iets wat in het ene FabLab is ontworpen kan heel makkelijk worden gemaakt in een ander FabLab, door de ontwerpbestanden op te sturen. Dit zorgt voor een internationale gemeenschap die samen dingen kan maken. Het idee is dat hiermee een opensource hardware platform ontstaat, zoals er nu ook een enorme opensource software gemeenschap bestaat.
\end{quote}(Van http://www.fablab.nl/articles/2007/05/22/de-machines/ )

Bij een standaard fablab zijn de volgende machines aanwezig:
\begin{itemize}
  \item Stickersnijder
  \item 3D frees apparaat
  \item 2D laser snijder
  \item 2D snijder groot formaat
\end{itemize}
Met deze machines kan je dus in principe bijna alles ontwerpen wat je maar wilt.
\section{Wat vind ik er van?}
Het gebruik van een fablab kan een hoop voordelen hebben wanneer je iets wilt ontwerpen maar niet het geld aanwezig is om bepaalde benodigdheden te kopen. Door bij een fablab te gaan werken kan je op een simpele manier aan dit product werken.
Of het echt zin heeft om bij een school te hebben weet ik niet. Een fablab is niet specifiek bedoeld voor scholen, maar voor iedereen die er is. Maar wanneer het in school echt aanwezig is ka niet iedereen zomaar naar binen waardoor het hele idee van fablab niet echt meer werkt.
De voordelen aan de andere kant zijn weer dat studenten heel makkelijk een project kunnen uitvoeren welke ontwerpen bevatten. De studenten moeten echter hierbij weer kennis hebben over de apparatuur welke in het fablab is.
