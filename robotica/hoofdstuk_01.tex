\chapter{Het project}
Aan het begin van het project kregen we de opdracht om met behulp van een bioliod een robot te bouwen. De opdracht was in eerste instantie om hiervan een vechtrobot te maken zodat er in de klas een ``gevecht'' gehouden kon worden tussen de diverse opdrachten. Wij wouden echter iets anders doen.

Een aantal jaar geleden hebben een groep studenten een robothond van een bioliod gebouwd. Deze hond wordt met regelmaat gebruikt op dingen als open dagen en proefstuderen. Helaas heeft deze hond een paar nadelen. Doordat bioliod veel gebruik maakt van schroefjes, valt de robothond met regelmaat uit elkaar. Een echte oplossing is hier niet voor, wanneer je namelijk de schroefjes vast gaat lijmen krijg je ze er niet meer in nadat ze alsnog los getrild zijn. Onze opdracht was dus eigen vrij simpel, we nemen de orginele Fluffy en gaan die compleet nabouwen. Hierbij zorgen we ervoor dat hij precies zo wordt gebouwd als de orginele fluffy. Op deze manier weten we zeker dat hij even goed werkt. We konden namelijk niet de orginele code aanpassen, doordat we deze niet meer hebben. We maken dus ook gebruik van het orginele blok met de code van fluffy. In principe is het dus de compleet zelfde hond, enkel geheel opnieuw opgebouwd.

\section{Open Source Robot Platform}
Naast het ontwerpen van de eigenlijke robot moest er ook een verslag geschreven worden over het opzetten van een Open Source Robot Platform met hierbij een onderzoek naar wat de huidige mogelijkheden zijn om een robot te maken. Dit onderzoek hebben wij samen met een andere groep gedaan, zodat de informatie welke we vonden tot een goed resultaat komt. 
