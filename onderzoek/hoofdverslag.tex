\hoofdstuk{Onderzoeksvraag}
Is een commercieel bedrijf geschikt om privacy gevoelige data op te slaan zoals bijvoorbeeld in het EPD staat opgeslagen? 

\paragraaf{Uitleg}
De laatste tijd is de overheid met regelmaat in het nieuws met problemen met veiligheid van ICT systemen en de privacy gevoeligheid hierachter. Hieruit blijkt met regelmaat dat de overheid niet al te slim omgaat met de problemen welke ondekt wordt en hierna afgehandeld. Een goed voorbeeld hierbij zijn de recente problemen bij diginator met de problemen met certificaten om websites te beveiligen. Maar niet alleen de overheid heeft problemen met veiligheid en privacy. Er zijn ook genoeg commerciele bedrijven welke niet veel hierom geven. Een aantal maanden geleden was in het nieuws dat de overheid het opslaan van de gegevens voor in het EPD uitbesteed had aan een commercieel bedrijf waarvan voor de meeste mensen niet bekend was of dit bedrijf hiermee wel correct omgaat. De overheid gaf hier ook geen openheid in, het gaf in eerste instantie niet eens openheid over dat het EPD daadwerkelijk was gehost bij een commercieel bedrijf. 
Het doel van dit onderzoek is om te kijken of er een commercieel bedrijf met winstoogmerk op zo'n manier kan werken dat het toch de privacy van de gebruikers veilig houd, maar toch ook aan de eigen bedrijfsdoelen kan houden. Hierbij wordt ook gekeken naar de manier van beveiligen van zowel de data opslag als de verbinding om toegang te krijgen tot de data voor artsen. Tevens wordt er gekeken naar alternatieve oplossing zonder het gebruik van een commercieel bedrijf.
