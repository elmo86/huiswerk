\hoofdstuk{Keuzes en deelvragen} 

Tijdens de uitvoering van dit haalbaarheidsonderzoek werd ik
geconfronteerd met een aantal deelvragen die ik moest beantwoorden en
een aantal keuzes die ik moest maken.

\paragraaf{Blackboard vs N@tschool} % N@tschool 

Eerst deelvraag was: `Is het niet beter om op BlackBoard leeromgeving
over te stappen? Veel universiteiten gebruiken deze leeromgeving al en
hebben daar redelijk goede ervaringen mee. Het grootste voordeel van
BlackBoard is dat MapleTA daarvoor al een plug-in. Voor N@tschool
bestaat zo'n plug-in niet.'

Het antwoord hierop kan heel kort zijn. Namelijk nee. Er zijn
inderdaad universiteiten die gebruik maken van BlackBoard, maar de \HR{}
maakt al jaren gebruik van N@tschool. En zelfs nu zijn er nog enkele
docenten die niet weten hoe ze ermee om moeten gaan. Als er nu weer
een nieuwe leeromgeving geïntroduceerd wordt, zou het weer heel veel
tijd en geld kosten om docenten ermee te leren werken. Bovendien
moeten er dan voor BlackBoard ook licenties gekocht worden.

Verder heb ik tijdens gesprekken met gebruikers van MapleTA
gecombineerd met BlackBoard gehoord dat het niet altijd even goed
werkt. Zo is onmogelijk om via BlackBoard sommige instellingen in
MapleTA te veranderen puur omdat je daar niet bij komt. Natuurlijk kan
je erbij komen door in MapleTA rechtstreeks in te loggen, maar dat is
natuurlijk niet de bedoeling. Ook zou BlackBoard tijdrovend zijn
tijdens het inlezen en importeren van grote groepen studenten, terwijl
in MapleTA rechtstreeks dat een kwestie van een paar milliseconden is.

Het is dus beter om het huidige systeem N@tschool aan MapleTA
te koppelen en docenten een workshop te geven om ze te leren met de
nieuwe koppeling om te gaan.

\paragraaf{`Single Sign On'}

Om te weten of het huidige systeem samen met MapleTA gebruikt zou
kunnen worden, was het belangrijk om te weten of MapleTA de Single
Sign On (SSO) ondersteunt. Oftewel kan MapleTA via LDAP van de \HR{} aan
het centrale loginpunt gekoppeld worden?

Dat MapleTA LDAP ondersteund was wel duidelijk, omdat ook de oude
versie van de toetsomgeving een link met LDAP had. Zo konden alleen
studenten van de \HR{} erop inloggen. Om uit te zoeken of het ook SSO
ondersteund, werd er een testomgeving gebruikt. Op de testserver werd
er volgens de handleidingen op de Confluence [3] wat veranderd aan het
web.xml bestanden en kwamen er wat andere bestanden bij. Daarna kreeg
MapleTA een geldige `toegangsticket' van de login systeem, en kon men
verder in MapleTA werken. Enige probleem dat eruit kwam is de
verkeerde configuratie van Tomcat tijdens de installatie van
MapleTA. Maar dit probleem werd al in het hoofdstuk hiervoor, in
paragraaf \ref{sec:proxy} op pagina \pageref{sec:proxy} beschreven.

\paragraaf{Plug-in of script}

De volgende deelvraag was: `Hoe kan de hoofdpagina van MapleTA
omgeleid worden? Moet daarvoor veel broncode veranderd worden of is
het mogelijk of noodzakelijk om een aparte plug-in te schrijven?'

Het uiteindelijke besluit was om in de broncode van MapleTA zelf een
aantal bestanden te veranderen en wat bestanden toe te voegen. Dit omdat ik
geen idee heb hoe ik een plug-in moet schrijven. Wat ik ervan
weet, is dat dit veel meer dan de geplande tijd in beslag zou
nemen. Tijdens het testen met de testserver was het gelukt om een
geldige `toegangsticket' te krijgen van het centrale loginpunt. Dus
directe integratie in de MapleTA files werkt.

\paragraaf{Cijferregistratie} 

Een andere belangrijke deelvraag was: `Hoe moet de cijferregistratie
gekoppeld worden? Moet er een aparte buffer komen waar de docenten
resultaten kunnen uploaden en het Osiris team deze kan downloaden, of
kan dat rechtstreeks vanuit MapleTA gebeuren, zonder enige
tussenruimte.'

Oorspronkelijk was het de bedoeling om een aparte server op te zetten
die als het ware een buffer zou moeten vormen tussen de docent en het
Osiris team. De docent zou de functionaliteit van MapleTA gebruiken en
de cijferlijst in een bestand met `comma seperated values' (csv-file)
te downloaden. Daarna zou er op een extra pictogram gedrukt moeten
worden om naar de bufferserver te gaan. Op deze server zou ook een
script draaien die de csv-file omzet naar het juiste formaat en
vervolgens in de database opslaat. De docent zou op de server met zijn
digitale handtekening moeten inloggen, het file uploaden en daarbij
een `volgnummer' krijgen. Verder moet de docent de cijferlijst ook
uitprinten, daarop het gekregen `volgnummer' opschrijven en dit bij
het Osiris team inleveren. Osiris medewerkers zouden de cijferlijst
daarna gewoon kunnen downloaden, en verder verwerken net alsof ze het
gewoon per mail zouden krijgen.

Later kwam er nog een mogelijkheid bij, die grotendeels overeenkomt
met de vorige oplossing, alleen een stukje simpeler. Nu ik toch heb
geleerd hoe je MapleTA bestanden kan wijzigen, verwijderen en
toevoegen, waarom dan een aparte server en database opzetten om het
datatype te veranderen? Het script voor het omzetten naar het juiste
formaat zou op de MapleTA server zelf kunnen staan. Om dat script te
activeren, zou er in de `gradebook' file een extra knop moeten komen,
genaamd `Export to Osiris'. Die zou de csv-file kunnen binnenhalen,
parsen, en aan de gebruiker vragen waar het nieuwe Excelbestand
opgeslagen moet worden. Daarna kan de docent een nieuwe sheet gewoon
volgens de oude procedure opsturen. Voor het Osiris team zou er zelfs
helemaal niets veranderen.

Keuze voor een oplossing in dit geval lag voor de hand. In samenspraak
met de begeleidende docent voor de tweede oplossing gekozen.

\paragraaf{HBO \& MBO} 

Nu er besloten is om MapleTA aan het centrale loginpunt te koppelen is
het vanzelfsprekend dat de MBO studenten er niet bij kunnen omdat ze
niet in de LDAP-server van de \HR{} zijn opgenomen. Het is dan ook
redelijk vanzelfsprekend dat de drie mogelijke oplossingen voor dit
probleem, die beschreven staan in paragraaf \ref{sec:MBO} op pagina
\pageref{sec:MBO}, niet reëel zijn, omdat ze al het werk die in de
navigatie mogelijkheden gestopt is, teniet doen. De enige mogelijke
oplossing blijft dan het opzetten van twee MapleTA servers. De tweede
MapleTA server zal niet samen met N@tschool kunnen werken, maar het is
dan wel bereikbaar buiten de \HR{} voor studenten aan het MBO. Het
belangrijkste is dat de tentamenserver wel gekoppeld is aan
N@tschool. Daarmee hoeven de studenten niet naar de juiste toets in
MapleTA zelf te zoeken, ze klikken gewoon op het linkje die de docent
bij zijn vak heeft neergezet, en ze zijn gelijk bij het juiste
tentamen. Zo gaat er geen kostbare tijd verloren aan zoeken.

\paragraaf{OS en andere belangrijke keuzes}

Verder is er in dit onderzoek besloten om de testserver onder Linux te
draaien en niet onder Windows. De reden hiervoor is dat de MapleTA
servers van de \HR{} Linux als besturingssysteem gebruiken. Daarmee is
het makkelijker een passende handleiding voor het overbrengen van
servers te schrijven.

Een andere keuze die gemaakt was, was het gebruik van Apache2 in
plaats van Lighttpd. Reden hiervoor was het feit dat ik wat meer
bekend ben met Apache2. Maar mocht het probleem dat de proxy oplevert
niet in de configuratie tussen apache en MapleTA zitten, dan zal deze
keuze heroverwogen worden en waar nodig zal er alsnog op Lighttpd
overgestapt kunnen worden.

