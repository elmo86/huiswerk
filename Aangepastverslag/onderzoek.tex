\hoofdstuk{Onderzoek}

In dit onderdeel wordt de context van het haalbaarheidsonderzoek
geschetst.

\paragraaf{Situatieschets}

De \HR{} (HR) maakt momenteel gebruik van N@tschool. Het
systeem is vergelijkbaar met BlackBoard welke op meeste universiteiten
wordt gebruikt. Volgens de website van Fontys Hogeschool
\cite{fontis}, die ook gebruikt maakt van N@tschool, is N@tschool:

\begin{quotation} een elektronische leeromgeving met uitgebreide
mogelijkheden voor de student, de docent en de organisatie.

Studenten kunnen binnen N@Tschool! digitale leerstof doorlopen,
toetsen maken en (samen)werken aan projecten. Ook is een Digitaal
Portfolio instrument aanwezig.

De digitale leerstof wordt aangeboden in de vorm van studieroutes, te
vergelijken met een reeks lessen van een vak. Een aantal studieroutes
samen vormt een studieprogramma. Het werken aan een studieroute kan
online plaatsvinden, maar de materialen kunnen ook gedownload worden
en offline worden gebruikt. Studieroutes begeleiden doet de docent
o.a. door het bekijken van en reageren op toetsresultaten.

Er is een aparte projectomgeving waarin een docent projecten kan
opstarten. De studenten krijgen daarmee de beschikking over een
werkruimte waarin ze kunnen samenwerken, documenten plaatsen en
uitwisselen en waarin versie en eigenaar van de documenten worden
bijgehouden. Daarnaast is er een procesruimte waar ze via een
projectthermometer hun mening kunnen geven over het verloop van het
project, waarin opdrachten en documenten beoordeeld kunnen worden en
waar de procesgang te volgen is.

Het Digitaal Portfolio bestaat uit een studiedossier,
publicatiedossier en een assessmentdossier. Het portfolio bevat
persoonlijke materialen van een student en kan op verschillende
manieren worden ingezet, bijvoorbeeld voor de begeleiding en/of
beoordeling van (delen van) het leerproces van een student door
docent, begeleider of medestudent (peer-begeleiding). Het
studiedossier bevat een dynamisch gedeelte (Onderhanden werk) waarin
de student van dag tot dag werkt en een statisch gedeelte (Afgerond)
van afgeronde en beoordeelde materialen die niet meer wijzigbaar
zijn. Met het publicatiedossier heeft de student de mogelijkheid eigen
werk via het Internet te publiceren en het assessmentdossier bevat
dossiers met bewijsmateriaal ten behoeve van toetsing en assessment.

Verder beschikt N@Tschool! over een Leermanagementsysteem
(LMS). Hierin worden opleidingscompetenties (of eindtermen,
vaardigheden, etc.) en studieprofielen vastgelegd. Studenten kunnen
vervolgens worden gekoppeld aan een studieprofiel, waardoor zij een
persoonlijk competentieprofiel krijgen (PCM). Aan de hand van dit
persoonlijke profiel kan de student relevante studieroutes kiezen of
zelfs geheel vraaggestuurd werken. Eventuele eerder verworven
competenties kunnen in het studieprofiel als behaald worden
geregistreerd en persoonlijke leerdoelen van een student kunnen worden
toegevoegd. In het LMS registreert een student zijn eigen
studievoortgang. Bewijzen hiervoor legt hij vast in het Portfolio. LMS
en Portfolio hebben een logisch verband met elkaar, maar kunnen ook
los van elkaar worden gebruikt.

In het systeem zit een toetsvoorziening waarbinnen toetsen en
toetsitems aangemaakt kunnen worden. Per toets kunnen tal van
instellingen worden gekozen om de omvang, vorm en afhandeling van de
toets vast te leggen.
\end{quotation}

Verder maakt de \HR{} gebruik van MapleTA. MapleTA is een systeem voor
het maken en afnemen van toetsen, opdrachten en oefeningen met behulp
van een html-browser zoals Firefox of IE. Antwoorden worden
automatisch nagekeken en beoordeeld met behulp van een module van het
mathematische softwarepakket Maple.

Het mooiste van MapleTA is dat één vraagstuk zo geprogrammeerd kan
worden dat er steeds net een andere vraag gesteld wordt, zo lang de
formule voor het berekenen maar equivalent is. Bijvoorbeeld de
variabelen lengte, breedte, oppervlakte en omtrek kunnen
gespecificeerd worden en de student wordt dan naar één van die
variabelen gevraagd, terwijl hij de andere twee krijgt. De vierde
variabele is voor de afwisseling.

Een voorziening van MapleTA is het bijhouden van de resultaten. Er
wordt niet alleen het cijfer bijgehouden maar ook alle vragen die de
student kreeg en zijn antwoorden erop. Als de student meerdere
toetsmogelijkheden heeft gehad, kan de docent bepalen welk cijfer
meegerekend moet worden. Het kan het laatste, beste of het gemiddelde
resultaat zijn.

Gezien het feit dat de cijferregistratie op de \HR{} in Osiris gebeurd
moet de docent uiteindelijk resultaten van alle studenten handmatig in
Osiris invullen.

\paragraaf{Doelstelling} 

Het zou mooi zijn als vanuit een bepaald vak in N@tschool een directe
koppeling naar de bijbehorende toets gemaakt zou kunnen
worden. Daarmee wordt het makkelijker voor studenten om de juiste
(oefen-)toets te vinden. Het voorkomt ook dat studenten zich per
ongeluk in de verkeerde klas of bij het verkeerde vak
inschrijven. Voor de docenten heeft het het voordeel dat ze duidelijk
in kunnen zien voor welk vak de toets bedoeld is. Verder kunnen er
makkelijker kleine oefentoetsen aan het lesmateriaal gekoppeld worden.
\paragraaf{Probleemstelling} Om de bovengenoemde doelstelling te
bereiken moet MapleTA aan het centrale loginpunt gekoppeld zijn. Dat
betekent dat de inlogpagina van MapleTA vervangen moet worden door
\url{http://login.hro.nl}. Daarmee zullen de toetsen na eenmalig
inloggen, rechtstreeks vanuit N@tschool bereikbaar zijn.

\paragraaf{Onderzoeksmethode}

Voor dit onderzoek heb ik de waterval methode \cite{waterval}
gebruikt, omdat het onderzoek ten eerste heel breed is en ten tweede
heel erg aan elkaar hangend. Voor de uitvoer werd de hoofdvraag in
steeds kleine deelvragen verdeeld. Pas wanneer er antwoord op de
eerste deelvraag werd gevonden, kon er naar de volgende deelvraag
overgegaan worden.

\paragraaf{Deelvragen}

Er zijn tijdens het onderzoek meer kleine vragen naar boven gekomen,
maar de belangrijkste deelvragen die van groot belang op de
projectvoortgang waren, zijn hieronder te lezen.

\begin{description}
\item[1.] Is het niet beter op BlackBoard leeromgeving over te
  stappen? Meeste universiteiten gebruiken deze leeromgeving al, en
  hebben daar redelijk goede ervaringen mee. Het grootste voordeel is
  dat MapleTA al een plug-in heeft die deze BlackBoard module
  ondersteunt. Voor N@tschool bestaat zulke plug-in niet.
\item[2.] Ondersteunt MapleTA de Single Sign On? Oftewel kan MapleTA
  via de LDAP van de HR aan het centrale loginpunt gekoppeld worden?
\item[3.] Hoe kan de hoofdpagina van MapleTA omgeleid worden?  Moet er
  veel broncode daarvoor veranderd worden of is het mogelijk om een
  aparte plug-in te schrijven?
\item[4.] Hoe moet de cijferregistratie gekoppeld worden? Moet er een
  aparte buffer komen waar de docenten resultaten kunnen uploaden en
  het Osiris team deze kan downloaden, of kan dat rechtstreeks vanuit
  MapleTA gebeuren, zonder enige tussenruimte.
\item[5.] Hoe zorgen we ervoor dat ook MBO studenten de oefentoetsen
  kunnen bereiken? Het is van groot belang omdat de HR studenten
  begeleid die willen doorstromen naar het HBO. Zo worden er
  verschillende bijspijkermodulen voor wiskunde gegeven en worden er
  keuzevakken zoals robotica gegeven.
\end{description}

\paragraaf{Planning} 

Tijdsindeling van dit onderzoek was niet erg vast gesteld. De enige
deadline was de inleverdatum van dit rapport. Helaas is die op 18
oktober (begin van de herfstvakantie) verlopen. Als gevolg van extra
activiteiten als studentassistent kon ik maar twee van de vier
geplande dagen per week aan het onderzoek besteden. Bovendien moest ik
tijdens het onderzoek met veel mensen spreken waarbij veel tijd
verloren ging aan wachten op een voor beide partijen gunstige
gelegenheid.
