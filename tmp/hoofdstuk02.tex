\chapter{Recursie}
\section{Opdracht 1}
Leidt de recurrente betrekking af voor het aantal verbindingslijnstukken tussen $n$ punten.

\section{Opdracht 2}
De volgende C-functie berekent de faculteit van een natuurlijk getal $n\geq 0$:
\begin{lstlisting}
long faculteit (int n)
{
  if (n == 0)
  {
    return 1;
  }
  else
  {
    return n*faculteit(n-1);
  }
}
\end{lstlisting}
Maak een iteratieve versie van deze functie.

\section{Opdracht 3}
Maak een recursief algortime voor de torens van Hanoi.

\section{Opdracht 4}
Bepaal het aantal stappen $T_n$ voor het verplaatsen van $n$ schijven bij de torens van Hanoi, indien rechtstreekse verplaatsingen van toren A naar toren C verboden zijn. Elke schijf moet langs toren B.

\section{Opdracht 5}
Bereken $alg\_a(n)$ en $alg\_b(n)$ voor $n=1\ldots 5$. Bereken de effici\"{e}ntie van algoritme $alg\_a$ en van algoritme $alg\_b$ in O-notatie:
\begin{itemize}
\item[(a)]\begin{lstlisting}
alg_a(n):resultaat
if n > 1  then
return (alg_a(n-1)+alg_a(n-1))
else
return (1)
\end{lstlisting}
\item[(b)]\begin{lstlisting}
alg_b(n): resultaat
if n > 1 then
return 2 * alg_b(n-1))
else
return 1
\end{lstlisting}
\end{itemize}

\section{Opdracht 7}
Leidt een recurrente betrekking af voor de berekening van een $x^p$, waarbij  $x$ een re\"{e}l getal en $p$ een natuurlijk getal van $n$ bits. Maak hiervan een recursief algoritme. Bepaal de tijdcomplexiteit voor dit algoritme op dezelfde processortype als die uit de vorige opgave.

\section{Opdracht 11}
Een ingewikkelde vorm van recursie is de functie van \emph{Ackermann}:

\begin{displaymath}
ack(m,n)=\left\{ 
\begin{array}{ll}
n+1 & als m = 0 en n \geq 0\\
ack(m-1,1) & als m < 0 en n = 0\\
Ack(m-1, ack(m,n-1)) & als m > 0 en n > 0\\
\end{array}
\right|
\end{displaymath}

Toon aan dat $Ack(2,3)=9$.
