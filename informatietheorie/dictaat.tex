\documentclass[a4paper,11pt]{report}
% Hier hebben we de preamble, alle document settings moeten hier:
\usepackage{graphicx}
\usepackage{url}
\usepackage{appendix}
\usepackage[titles]{tocloft}
\usepackage[dutch]{babel}
\usepackage{listings}
\usepackage{makeidx}
\usepackage{float}
\usepackage[hypertexnames=false]{hyperref}

% Custom LaTeX commands:
% ~ == {\raise.17ex\hbox{$\scriptstyle\sim$}}
\newcommand{\customtilde}{\raise.17ex\hbox{$\scriptstyle\sim$}}
% Meta info pdf:
\hypersetup{
%bookmarks=true, % show bookmarks bar?
unicode=false, % non-Latin characters in Acrobat’s bookmarks
pdftoolbar=true, % show Acrobat’s toolbar?
pdfmenubar=false, % show Acrobat’s menu?
pdffitwindow=false, % window fit to page when opened
%pdfstartview={FitH}, % fits the width of the page to the window
pdftitle={Informatietheorie}, % title
pdfauthor={Paul Sohier}, % author
pdfsubject={Informatietheorie}, % subject of the document
pdfcreator={make}, % creator of the document
pdfproducer={make}, % producer of the document
pdfkeywords={Linux} {Basis}, % list of keywords
pdfnewwindow=true, % links in new window
colorlinks=false, % false: boxed links; true: colored links
linkcolor=black, % color of internal links
citecolor=green, % color of links to bibliography
filecolor=magenta, % color of file links
urlcolor=cyan % color of external links
}

% Paragrafen hebben een witregel ertussen, en geen indent tab:
\setlength{\parindent}{0.0in}
\setlength{\parskip}{0.1in}
% Onderstaande is voor de dots tussen chapter title + blz. 
\makeatletter
\renewcommand*\l@chapter[2]{%
  \ifnum \c@tocdepth >\m@ne
    \addpenalty{-\@highpenalty}%
    \vskip 1.0em \@plus\p@
    \setlength\@tempdima{1.5em}%
    \begingroup
      \parindent \z@ \rightskip \@pnumwidth
      \parfillskip -\@pnumwidth
      \leavevmode \bfseries
      \advance\leftskip\@tempdima
      \hskip -\leftskip
      #1\nobreak\normalfont\leaders\hbox{$\m@th
        \mkern \@dotsep mu\hbox{.}\mkern \@dotsep
        mu$}\hfill\nobreak\hb@xt@\@pnumwidth{\hss #2}\par
      \penalty\@highpenalty
    \endgroup
  \fi}
\makeatother

% End of title + blz.
% Table of content depth van 4, dus tm paragraph
\setcounter{tocdepth}{4}
%\renewcommand{\baselinestretch}{1.5} 1.5 regelafstand 

%Pas listings aan zodat ze duidelijker zijn
\lstset{ %
  language=bash,                % choose the language of the code
  basicstyle=\footnotesize,       % the size of the fonts that are used for the code
  numbers=left,                   % where to put the line-numbers
  numberstyle=\footnotesize,      % the size of the fonts that are used for the line-numbers
  numbersep=5pt,                  % how far the line-numbers are from the code
  showspaces=false,               % show spaces adding particular underscores
  showstringspaces=false,         % underline spaces within strings
  showtabs=false,                 % show tabs within strings adding particular underscores
  frame=lr,	                % adds left and right lines
  tabsize=2,	                % sets default tabsize to 2 spaces
  captionpos=b,                   % sets the caption-position to bottom
  breaklines=true,                % sets automatic line breaking
  breakatwhitespace=false,        % sets if automatic breaks should only happen at whitespace
%  escapeinside={\%*}{*)},         % if you want to add a comment within your code
  morekeywords={*,...}            % if you want to add more keywords to the set
}
%hyperref aanpassingen
\hypersetup{pdfborder=0 0 0}
% We gebruiken de index
\makeindex

% Einde preamble, begin document; 
\begin{document}
% Title page
\title{
  Informatie theorie
}
\author{
  Sebastiaan Polderman\\
  nummer
  \and
  Paul Sohier\\
  0806122
}
\date{\today}
% Print de title
\maketitle

% Abstract (+ in toc)?:
\begin{abstract}\centering

\end{abstract}

% De table of contents + toc in toc:
\tableofcontents
\addcontentsline{toc}{chapter}{\numberline{}Inhoudsopgave}

% HACK: Page number fixen. Dit is voor makeidx + hyperref
% 15 is de pagina met hoofdstuk 01, dus inleiding erop. 
% Dit zorgt ervoor dat alle
\setcounter{page}{3}
% Nu kunnen we de losse hoofdstukken gaan includen. 
% Includen gebeurt met basename, dus zonder .tex
\chapter{Het project}
Aan het begin van het project kregen we de opdracht om met behulp van een bioliod een robot te bouwen. De opdracht was in eerste instantie om hiervan een vechtrobot te maken zodat er in de klas een ``gevecht'' gehouden kon worden tussen de diverse opdrachten. Wij wouden echter iets anders doen.

Een aantal jaar geleden hebben een groep studenten een robothond van een bioliod gebouwd. Deze hond wordt met regelmaat gebruikt op dingen als open dagen en proefstuderen. Helaas heeft deze hond een paar nadelen. Doordat bioliod veel gebruik maakt van schroefjes, valt de robothond met regelmaat uit elkaar. Een echte oplossing is hier niet voor, wanneer je namelijk de schroefjes vast gaat lijmen krijg je ze er niet meer in nadat ze alsnog los getrild zijn. Onze opdracht was dus eigen vrij simpel, we nemen de orginele Fluffy en gaan die compleet nabouwen. Hierbij zorgen we ervoor dat hij precies zo wordt gebouwd als de orginele fluffy. Op deze manier weten we zeker dat hij even goed werkt. We konden namelijk niet de orginele code aanpassen, doordat we deze niet meer hebben. We maken dus ook gebruik van het orginele blok met de code van fluffy. In principe is het dus de compleet zelfde hond, enkel geheel opnieuw opgebouwd.

\section{Open Source Robot Platform}
Naast het ontwerpen van de eigenlijke robot moest er ook een verslag geschreven worden over het opzetten van een Open Source Robot Platform met hierbij een onderzoek naar wat de huidige mogelijkheden zijn om een robot te maken. Dit onderzoek hebben wij samen met een andere groep gedaan, zodat de informatie welke we vonden tot een goed resultaat komt. 

Het doel van ons onderzoek naar het OSRP was om te kijken wat er momenteel voor opties waren wanneer je gebruik wou maken van een robot, en hoe je dit zo kon combineren tot een beter platform. Hierbij hebben wij dus voornamelijk theoretische gewerkt, met als eindresultaat een compleet onderzoek van wat wij denken dat goed is als nieuw platform.

Het onderzoek over het Open Source Robot Platform is te vinden in het andere document.

\chapter{De bouw}
Doordat we geen compleet eigen ontwerp gingen maken van Fluffy hebben we eerst helemaal uitgezocht hoe Fluffy precies in elkaar zit. Zonder deze informatie kunnen we hem niet namaken, en doordat we de code niet hebben moeten we er ook voor zorgen dat hij ook goed in elkaar zit zoals de orginele Fluffy. 

Het belangrijkste hiervan zijn de motoren. De motoren worden aangestuurd via het nummer van de motor. Iedere motor heeft een uniek nummer in de robot. Via dit nummer wordt die motor aangestuurd. Wanneer de motor opeens op een andere plek zou zitten als in de orginele Fluffy gaat hij mischien wel lopen in plaats van met zijn staart te kwispellen. En dit willen we uiteraard niet zien gebeuren. 
Naast de motoren moesten we er ook voor zorgen dat alle plastic onderdelen die verder gebruikt zijn op dezelfde manier erop komen. Anders heb je mogelijk verschil in grote van poten, waardoor hij bijvoorbeeld niet meer goed loopt. 

Het hele project lijkt op papier veel minder werk als een normaal project, maar doordat we hem precies moeten namaken heeft dit project meer tijd qua onderzoek gekost als een normaal project. Hiernaast moesten we ook iedere keer controleren of wat we gedaan hadden in dat stukje van Fluffy ook wel klopten met wat er in het orgineel zat. En wanneer dit niet het geval was (Wat zo af en toe wel eens voor kwam), moest dit weer uit elkaar gehaald worden en opnieuw bevestigd. Door dit soort kleine dingen duurt dit project vrij lang.

\hoofdstuk{RS485}
RS485 is een communicatie protocol wat ons in staat stelt om zonder grote wijzingen in de bedrading bij het toevoegen van een nieuw kastje gebruik te maken van stabiele communicatie. Het protocol staat over een grote lengte toe om te communiceren met een groot aantal nodes. Mocht de lengte tussen de nodes te groot zijn kan er indien nodig gebruik gemaakt worden van een RS485 repeater welke het signaal herhaalt.

RS485 maakt standaard gebruik van half duplex. Hierbij zijn er twee draden nodig voor de communicatie en \'{e}\'{e}n draad voor de common ground. Full duplex is ook mogelijk met RS485, maar wij maken er in ons geval geen gebruik van.

Het nadeel van het gebruik van half duplex is dat er niet tegelijk verzonden en geluisterd kan worden. Hierdoor moet er dus aan de kant van de kastjes gecontroleerd worden of er niet gezonden wordt voordat het kastje gaat zenden. Wanneer een kastje gaat zenden terwijl een ander kastje aan het zenden is zal de communcatie niet goed verlopen. Door gebruik te maken van een centrale master (De computer), die aangeeft wie er mag zenden kan dit probleem verholpen worden. Dit heeft wel als nadeel dat er meer data nodig is om over te sturen omdat de master toestemming moet geven om te sturen. Echter, op korte afstand is de bandbreedte van RS485 tot 10MB/s (Tot 10 meter, hierna zal de snelheid afnemen), terwijl de data welke wij sturen vrij minimaal is.

In het schema hieronder is schematische te zien hoe de communcatie gaat vanuit de microcontroller naar de computer.

\begin{center}
  \figuur{scale=0.45}{plaatjes/scheme_rs485.png}{PNG}{RS485}
\end{center}


% Einde document
\end{document}
