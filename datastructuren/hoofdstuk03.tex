\chapter{De grafentheorie}
\section{opgave 1}
Een probleem, dat voor het eerst geformuleerd werd door een Chinese wiskundige, luidt: Een \emph{Chinese postbode} moet lopend de post bezorgen. Langs elke weg (een tak met een positieve afstandswaarde) staan brievenbussen. De optimale route heeft de minimale afstand. In wele type grafen is een optimale oplossing aanwezig?

\section{Opgave 2}
Een handelsreiziger moet vanaf een basis een aantal steden bezoeken met de kortste reisafstand en daarna terugkeren op zijn thuisbasis. In welk type grafen is een optimale oplossing aanwezig?

\section{Opgave 3}
Kan de volgende graaf zonder sijnende lijnen getekend worden op een plat vlak?
\\
\setlength{\unitlength}{1mm}
\begin{picture}(6,6)
  \put(0,0){\circle*{1}}
  \put(0,6){\circle*{1}}
  \put(6,0){\circle*{1}}
  \put(6,6){\circle*{1}}
  \put(0,0){\line(6,0){6}}
  \put(0,0){\line(0,6){6}}
%  \put(0,0){\line(6,6){6}}
  \put(6,0){\line(0,6){6}}
  \put(0,6){\line(6,0){6}}

  \put(0,0){\line(1,1){6}}
  \put(0,6){\line(1,-1){6}}
\end{picture}

\section{Opdracht 6}
In een graaf zijn vaak meer dan \"{e}\"{e}n opspannende boom te vinden. Bepaal van de graaf uit vraagstuk 4 het aantal opspannende bomen.

\section{Opdracht 7}
Bewijs dat een opspannende boom in een samenhangende graaf met $n$ knopen en $m$ takken uit $n-1$ takken bestaat.

\section{Opdracht 9}
Wat stelt de rij en de kolomsom van een adjaceentiematrix van een gerichte graaf voor?

