\documentclass[a4paper,11pt]{report}
% Hier hebben we de preamble, alle document settings moeten hier:
\usepackage{graphicx}
\usepackage{url}
\usepackage{appendix}
\usepackage[titles]{tocloft}
\usepackage[dutch]{babel}
\usepackage[xindy,style=altlistgroup]{glossaries}
\usepackage{listings}
\usepackage{makeidx}
\usepackage{fullpage}
%\usepackage{hyperref}
% Paragrafen hebben een witregel ertussen, en geen indent tab:
\setlength{\parindent}{0.0in}
\setlength{\parskip}{0.1in}
%% Woordenlijst moet aan het begin worden geinclude:
\loadglsentries{woordenlijst}
% Onderstaande is voor de dots tussen chapter title + blz. 
\makeatletter
\renewcommand*\l@chapter[2]{%
  \ifnum \c@tocdepth >\m@ne
    \addpenalty{-\@highpenalty}%
    \vskip 1.0em \@plus\p@
    \setlength\@tempdima{1.5em}%
    \begingroup
      \parindent \z@ \rightskip \@pnumwidth
      \parfillskip -\@pnumwidth
      \leavevmode \bfseries
      \advance\leftskip\@tempdima
      \hskip -\leftskip
      #1\nobreak\normalfont\leaders\hbox{$\m@th
        \mkern \@dotsep mu\hbox{.}\mkern \@dotsep
        mu$}\hfill\nobreak\hb@xt@\@pnumwidth{\hss #2}\par
      \penalty\@highpenalty
    \endgroup
  \fi}
\makeatother

% End of title + blz.
% Table of content depth van 4, dus tm paragraph
\setcounter{tocdepth}{4}
%\renewcommand{\baselinestretch}{1.5} 1.5 regelafstand 

%Pas listings aan zodat ze duidelijker zijn
\lstset{ %
  language=bash,                % choose the language of the code
  basicstyle=\footnotesize,       % the size of the fonts that are used for the code
  numbers=left,                   % where to put the line-numbers
  numberstyle=\footnotesize,      % the size of the fonts that are used for the line-numbers
  numbersep=5pt,                  % how far the line-numbers are from the code
  showspaces=false,               % show spaces adding particular underscores
  showstringspaces=false,         % underline spaces within strings
  showtabs=false,                 % show tabs within strings adding particular underscores
  frame=lr,	                % adds left and right lines
  tabsize=2,	                % sets default tabsize to 2 spaces
  captionpos=b,                   % sets the caption-position to bottom
  breaklines=true,                % sets automatic line breaking
  breakatwhitespace=false,        % sets if automatic breaks should only happen at whitespace
%  escapeinside={\%*}{*)},         % if you want to add a comment within your code
  morekeywords={*,...}            % if you want to add more keywords to the set
}
\makeindex

% Einde preamble, begin document:
\begin{document}
% Front page:
\title{
  Vakcode\\
  Datastructuren 2
}
\author{
  Paul Sohier\\
  0806122 \\
  \and
  Kees Jan Simon \\
  0791114 
}
\date{\today}
\maketitle
% Abstract. Heel kort wat het is:
%\begin{abstract}\centering
%Samenvatting
%\end{abstract}

% Nu een voorwoordje
%\chapter{Voorwoord}
%Hier een prachtig voorwoord. 

% De table of contents:
\tableofcontents

% Inleiding doen we ook nog in de master file:
%\chapter{Inleiding}
%Inleiding van het dictaat. Blablabla natuurlijk. 
% Einde inleiding

% Nu kunnen we de losse hoofdstukken gaan includen. 
% Includen gebeurt met basename, dus zonder .tex
\chapter{De effici\"{e}ntie van programmatuur}
\section{Opdracht 1}
Bepaal de herhalingsfrequentie $T(n)$ van:
  \begin{itemize}
    \item \texttt{for(i=n-1; i<n; i++)\{\}}
    \item \texttt{for(i=n-1; i <n\^2; i++)\{\}}
    \item \texttt{for(i=n; i<n; i++)\{\}}
    \item \texttt{i=0; while(i < n) \{a[i] = 0\}}
  \end{itemize}

\section{Opdracht 2}
Bepaal de O-notatie van:
\begin{itemize}
  \item $T(n)=17n^3-13n^2+10n+2000$
  \item $T(N)=3^n-13n$
  \item $T(n)=20log_2n+n^2$
\end{itemize}

\section{Opdracht 3}
In het sorteeralgoritme \emph{SelectionSort} worden de elementen in een lijst $a[1\ldots{}n]$ verwisseld van plaats afhankelijk van het onderlinge verschil in waarde. Het algoritme begint bij het eerste element te verwisselen met het kleinste element in de rest van de lijst. Vervolgens wordt het tweede elelment verwisseld met het ena kleinste element in de lijst, etc.:
\begin{lstlisting}
void SelectionSort(lijst a){
int i,j,k;
i = 0;
while(i < n){
  j = ++i;
  k = j;
  while (j <= n){
    if (a[j]<a[k]) k = j;
    ++j;
  }
  verwissel(a; i; k);
}
}
\end{lstlisting}
Het soorteeralgoritme \emph{InsertionSort} werkt de lijst $a[1\ldots n]$ door, het eerste stuk $(1\ldots i)$ is gesorteerd, het tweede stuk $(i+1\ldots n$ is nog ongesorteerd.
 Elk element $i+1$ uit het ongesorteerde lijstgedeelte wordt in het gesorteerde gedeelte tussengevoegd, waarna het gesorteerde gedeelte van de lijst met \'{e}\'{e}n element is toegenomen, ten koste vand e ongesorteerde deellijst:
\begin{lstlisting}
void InsertSort(lijst a){
int i, j, ready;
i = 1;
while(i<n){
  j = i++;
  ready = 0;
  while((j >=1)&&(ready==0)){
    if (a[j+1]<a[j]){
      verwissel(a;j+1;j);
      --j;
    }
    ready=1;
  }
}
\end{lstlisting}
Bepaal van \emph{InsertionSort} en \emph{SelectionSort} de effici\"{e}ntie in P-notatie.

\section{Opdracht 4}
\emph{BubbleSort} maakt gebruik van herhaald verwisselen van buurelementen in een lijst. Een element wordt naar voren verplaatst indien het kleiner is dan het buurelement. Geef de tijdcomplexiteit in O-notatie van het \emph{BubbleSort} algoritme. Is dit slechter dan de complexiteit van \emph{SelectionSort} en \emph{InsertionSort}?

\section{Opdracht 7}
Een \emph{polynoom} $f(x)\sum_{n}^{i=0} a_ix^j$ ($a0\ldots a_n$) zijn co\"{e}ffici\"{e}nten) wordt meestal uitgeschreven als:
\begin{displaymath}
f(x)=a_nx^n+a_{n-1}x^{n-1}+a_{n-2}x^{n-2}+\ldots +a_1x^1+a0
\end{displaymath}
Wij kunnen de polynoon $f(x)$ herschrijven met de \emph{regel van Horner}:
\begin{displaymath}
  f(x)=((\ldots((a_nx+a_{n-a})x+a_{n-2})x\ldots)x+a1)x+a0
\end{displaymath}


\chapter{Recursie}
\section{Opdracht 1}
Leidt de recurrente betrekking af voor het aantal verbindingslijnstukken tussen $n$ punten.

\section{Opdracht 2}
De volgende C-functie berekent de faculteit van een natuurlijk getal $n\geq 0$:
\begin{lstlisting}
long faculteit (int n)
{
  if (n == 0)
  {
    return 1;
  }
  else
  {
    return n*faculteit(n-1);
  }
}
\end{lstlisting}
Maak een iteratieve versie van deze functie.

\section{Opdracht 3}
Maak een recursief algortime voor de torens van Hanoi.

\section{Opdracht 4}
Bepaal het aantal stappen $T_n$ voor het verplaatsen van $n$ schijven bij de torens van Hanoi, indien rechtstreekse verplaatsingen van toren A naar toren C verboden zijn. Elke schijf moet langs toren B.

\section{Opdracht 5}
Bereken $alg\_a(n)$ en $alg\_b(n)$ voor $n=1\ldots 5$. Bereken de effici\"{e}ntie van algoritme $alg\_a$ en van algoritme $alg\_b$ in O-notatie:
\begin{itemize}
\item[(a)]\begin{lstlisting}
alg_a(n):resultaat
if n > 1  then
return (alg_a(n-1)+alg_a(n-1))
else
return (1)
\end{lstlisting}
\item[(b)]\begin{lstlisting}
alg_b(n): resultaat
if n > 1 then
return 2 * alg_b(n-1))
else
return 1
\end{lstlisting}
\end{itemize}

\section{Opdracht 7}
Leidt een recurrente betrekking af voor de berekening van een $x^p$, waarbij  $x$ een re\"{e}l getal en $p$ een natuurlijk getal van $n$ bits. Maak hiervan een recursief algoritme. Bepaal de tijdcomplexiteit voor dit algoritme op dezelfde processortype als die uit de vorige opgave.

\section{Opdracht 11}
Een ingewikkelde vorm van recursie is de functie van \emph{Ackermann}:

\begin{displaymath}
ack(m,n)=\left\{ 
\begin{array}{ll}
n+1 & als m = 0 en n \geq 0\\
ack(m-1,1) & als m < 0 en n = 0\\
Ack(m-1, ack(m,n-1)) & als m > 0 en n > 0\\
\end{array}
\right|
\end{displaymath}

Toon aan dat $Ack(2,3)=9$.

\chapter{De grafentheorie}
\section{Opdracht 1}
Een probleem, dat voor het eerst geformuleerd werd door een Chinese wiskundige, luidt: Een \emph{Chinese postbode} moet lopend de post bezorgen. Langs elke weg (een tak met een positieve afstandswaarde) staan brievenbussen. De optimale route heeft de minimale afstand. In wele type grafen is een optimale oplossing aanwezig?

\section{Opdracht 2}
Een handelsreiziger moet vanaf een basis een aantal steden bezoeken met de kortste reisafstand en daarna terugkeren op zijn thuisbasis. In welk type grafen is een optimale oplossing aanwezig?

\section{Opdracht 3}
Kan de volgende graaf zonder sijnende lijnen getekend worden op een plat vlak?
\\
\setlength{\unitlength}{1mm}
\begin{picture}(6,6)
  \put(0,0){\circle*{1}}
  \put(0,6){\circle*{1}}
  \put(6,0){\circle*{1}}
  \put(6,6){\circle*{1}}
  \put(0,0){\line(6,0){6}}
  \put(0,0){\line(0,6){6}}
%  \put(0,0){\line(6,6){6}}
  \put(6,0){\line(0,6){6}}
  \put(0,6){\line(6,0){6}}

  \put(0,0){\line(1,1){6}}
  \put(0,6){\line(1,-1){6}}
\end{picture}

\section{Opdracht 6}
In een graaf zijn vaak meer dan \"{e}\"{e}n opspannende boom te vinden. Bepaal van de graaf uit vraagstuk 4 het aantal opspannende bomen.

\section{Opdracht 7}
Bewijs dat een opspannende boom in een samenhangende graaf met $n$ knopen en $m$ takken uit $n-1$ takken bestaat.

\section{Opdracht 9}
Wat stelt de rij en de kolomsom van een adjaceentiematrix van een gerichte graaf voor?


%\chapter{Graafalgoritmen}
\section{Opdracht 1}
Een touringcarbedrijf wil een lucratieve toeritische route uitzetten langs een maximale opspannende boom in een netwerk. Geef een algoritme voor een maximum opspannende boom.


% Include van de biblio file, ook in de toc:
%\addcontentsline{toc}{chapter}{\numberline{}Bibliografie}
%\begin{thebibliography}{99}

\bibitem{bib.wikipedia}
Wikipedia, \textsl{De vrije encyclopedie}
\\\mbox{}\hfill\url{http://nl.wikipedia.org/}

\end{thebibliography}

% Verklarende woordenlijst + toc:
%\addcontentsline{toc}{chapter}{\numberline{}Verklarende Woordenlijst}
%\printglossaries

% Appendix:
%\appendix
%\chapter{Deel 1}\label{app.deel1}


%\addcontentsline{toc}{chapter}{Index}
%\printindex

% Einde document
\end{document}
