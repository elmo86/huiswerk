\chapter{Recursie}
\section{Opdracht 1}
Leidt de recurrente betrekking af voor het aantal verbindingslijnstukken tussen $n$ punten.

\texttt{antwoord:}

Iedere $n$de stap levert $n-1$ extra lijnen op bij het geheel, dus: $T(n) = T(n-1) + n-1$.

\section{Opdracht 2}
De volgende C-functie berekent de faculteit van een natuurlijk getal $n\geq 0$:
\begin{lstlisting}
long faculteit (int n)
{
  if (n == 0)
  {
    return 1;
  }
  else
  {
    return n*faculteit(n-1);
  }
}
\end{lstlisting}
Maak een iteratieve versie van deze functie.

\texttt{antwoord:}

\begin{lstlisting}
long int faculteit(int n) 
{
   if (n == 0) 
   {
      return 1;
   }
   else if (n > 0) 
   {
      int i;
      for(i = n-1;  i > 1; i--) 
      {
         n = n*i;
      }
      return n;
   }
   else 
   {
      return -1;
   }
}
\end{lstlisting}

\section{Opdracht 3}
Maak een recursief algortime voor de torens van Hanoi.

\texttt{antwoord:}

\begin{lstlisting}
int hanoi() 
{
   if(n > 0) 
   {
     return hanoi((n-1) + hanoi(n-1) + 1);
   } 
   else 
   {
     return n;
   }
}
\end{lstlisting}

\section{Opdracht 4}
Bepaal het aantal stappen $T_n$ voor het verplaatsen van $n$ schijven bij de torens van Hanoi, indien rechtstreekse verplaatsingen van toren A naar toren C verboden zijn. Elke schijf moet langs toren B.

\texttt{antwoord:}

\begin{lstlisting}
int hanoi(int n) 
{
   if(n == 1) {
     return 2;
   } 
   else 
   {
     return hanoi((n-1) + 2*pow(2,n) - 1);
   }
}
\end{lstlisting}

\section{Opdracht 5}
Bereken $alg\_a(n)$ en $alg\_b(n)$ voor $n=1\ldots 5$. Bereken de effici\"{e}ntie van algoritme $alg\_a$ en van algoritme $alg\_b$ in O-notatie:
\begin{itemize}
\item[(a)]\begin{lstlisting}
alg_a(n):resultaat
if n > 1  then
return (alg_a(n-1)+alg_a(n-1))
else
return (1)
\end{lstlisting}
\item[(b)]\begin{lstlisting}
alg_b(n): resultaat
if n > 1 then
return 2 * alg_b(n-1))
else
return 1
\end{lstlisting}
\end{itemize}

\texttt{antwoord:}

$n = 1\ldots{}5$ $->$ $\{1, 2, 4, 8, 16\}$ voor beide algoritmes.
Is het niet gewoon simpeler, dit:
$O(n)$ per functieaanroep.
twee keer per aanroep, recursief dus $O(2^n)$, \'{e}\'{e}n keer per aanroep, resultaat vermenigvuldigen, dus $ 2\cdot O(n) = O(n)$

\section{Opdracht 6}
Leidt een recurrente betrekking af voor een vermenigvuldiging van twee $n$-bit getallen $x$ en $y$. Maak hiervan een recursief algoritme. Bepaal tevens de tijdcomplexiteit van dit algoritme. Wij mogen veronderstellen dat voor het berokken proccessortype, het vermenigvuldigen en delen $O(n^2)$ instructies zijn. Daarentegen zijn optellen, aftrekken en 1-bits schuifacties $O(n)$ instructies. Het testen of een natuurlijk getal even-of oneneven is, zoals alle overige instructies atomair $O(1)$

\texttt{antwoord:}

\begin{lstlisting}
int mul(int x, int y) 
{
   if (y == 1) 
   {
       return x;
   } 
   else if ((y % 2) == 0) 
   {
       x = x << 1;
       return mul(x, (y >> 1));
   } 
   else 
   {
       return x + mul(x, (y - 1));
   }
}
\end{lstlisting}
Geen vermenigvuldigingen, alleen optellingen, dus $O(n)$

\section{Opdracht 7}
Leidt een recurrente betrekking af voor de berekening van een $x^p$, waarbij  $x$ een re\"{e}l getal en $p$ een natuurlijk getal van $n$ bits. Maak hiervan een recursief algoritme. Bepaal de tijdcomplexiteit voor dit algoritme op dezelfde processortype als die uit de vorige opgave.

\texttt{antwoord:}

\begin{lstlisting}
int pow(int x, int y) 
{
   if(y == 0) 
   {
      return 1;
   } 
   else if (y == 1) 
   {
      return x;
   } 
   else if ((y%2) == 0) 
   { //even getal
      int t = pow(x, (y >> 1));
      return mul(t,t);
   } 
   else 
   { //oneven
      return mul(x, pow(x, (y-1)));
   }
}
\end{lstlisting}
$O(n)$ ? We hebben alleen optellen/aftrekken/bitshifts zijn $O(n)$, compares zijn $O(1)$
